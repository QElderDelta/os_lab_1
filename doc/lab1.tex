\documentclass[a4paper, 12pt]{article}
\usepackage{cmap}
\usepackage[12pt]{extsizes}			
\usepackage{mathtext} 				
\usepackage[T2A]{fontenc}			
\usepackage[utf8]{inputenc}			
\usepackage[english,russian]{babel}
\usepackage{setspace}
\singlespacing
\usepackage{amsmath,amsfonts,amssymb,amsthm,mathtools}
\usepackage{fancyhdr}
\usepackage{soulutf8}
\usepackage{euscript}
\usepackage{mathrsfs}
\usepackage{listings}
\pagestyle{fancy}
\usepackage{indentfirst}
\usepackage[top=10mm]{geometry}
\rhead{}
\lhead{}
\renewcommand{\headrulewidth}{0mm}
\usepackage{tocloft}
\renewcommand{\cftsecleader}{\cftdotfill{\cftdotsep}}
\usepackage[dvipsnames]{xcolor}

\lstdefinestyle{mystyle}{ 
	keywordstyle=\color{OliveGreen},
	numberstyle=\tiny\color{Gray},
	stringstyle=\color{BurntOrange},
	basicstyle=\ttfamily\footnotesize,
	breakatwhitespace=false,         
	breaklines=true,                 
	captionpos=b,                    
	keepspaces=true,                 
	numbers=left,                    
	numbersep=5pt,                  
	showspaces=false,                
	showstringspaces=false,
	showtabs=false,                  
	tabsize=2
}

\lstset{style=mystyle}

\begin{document}
\thispagestyle{empty}	
\begin{center}
	Московский авиационный институт
	
	(Национальный исследовательский университет)
	
	Факультет "Информационные технологии и прикладная математика"
	
	Кафедра "Вычислительная математика и программирование"
	
\end{center}
\vspace{40ex}
\begin{center}
	\textbf{\large{Лабораторная работа №1 по курсу\linebreak \textquotedblleft Операционные системы\textquotedblright}}
\end{center}
\vspace{35ex}
\begin{flushright}
	\textit{Студент: } Живалев Е.А.
	
	\vspace{2ex}
	\textit{Группа: } М8О-206Б
	
	\vspace{2ex}
	\textit{Преподаватель: } Соколов А.А.
	
	
	\vspace{2ex}
	\textit{Оценка: } \underline{\quad\quad\quad\quad\quad\quad}
	
	 \vspace{2ex}
	\textit{Дата: } \underline{\quad\quad\quad\quad\quad\quad}
	
	\vspace{2ex}
	\textit{Подпись: } \underline{\quad\quad\quad\quad\quad\quad}
	
\end{flushright}

\vspace{5ex}

\begin{vfill}
	\begin{center}
		Москва, 2019
	\end{center}	
\end{vfill}
\newpage

\section{Задание}

Продемонстрировать работу с утилитой strace на примере одной из лабораторных работ.

Программе на вход поступает число n - количество процессов, которое будет создано в ходе выполнения и программы, и n названий файлов, в которые эти процессы будут записывать данные. Родительский процесс создает n дочерних процессов, которым поочередно передает символы с входной строки. Дочерний процесс записывает полученные символ в переданный ему файл.

В ходе выполнения лабораторной работы были использованы следующие системные вызовы:

\begin{itemize}
	\item open - открытие файла и получение его дескриптора
	\item read - использовался для чтения данных с входной строки
	\item sem\_open - создание семафора или открытие ранее созданного
	\item sem\_post - увеличение счетчика семафора
	\item sem\_wait - уменьшение счетчика семафора
	\item fork - создание дочерних процессов
	\item mmap - отображение файла на память
\end{itemize}
\section{Описание работы программы}

С помощью функции readLine, которая, используя read считывает одну строку с входной строки, программа получает количество процессов(число n), которые необходимо создать и имена файлов, в которые должна производиться запись. Затем создаётся n пайпов и n процессов. Затем в родительском процессе в цикле с условием, что возвращенное функцией read число больше 0 происходит запись считанного символа в определенный элемент массива char, который был записан в файл, отображенный в память, и увеличивает значение семафора, связанного с чтением. В то же время дочерний процесс считывает символ и записывает его в свой файл, а также увеличивает значение семафора, связанного с записью символов, тем самым давая родительскому процессу понять, что можно записать новый символ.

\newpage

\section{Исходный код}

\textbf{\large{main.c}}
\lstinputlisting[language=C]{../src/main.c}
\newpage
\section{Консоль}

\begin{verbatim}
	qelderdelta@qelderdelta-UX331UA:~/Study/OS/os_lab_4/src$ ./a.out
	3
	1.txt
	2.txt
	3.txt
	ttthhhiiisss   ssshhhooouuulllddd   bbbeee   iiinnn   fffiiillleee123                            
	qelderdelta@qelderdelta-UX331UA:~/Study/OS/os_lab_4/src$ cat 1.txt
	this should be in file1
	qelderdelta@qelderdelta-UX331UA:~/Study/OS/os_lab_4/src$ cat 2.txt
	this should be in file2
	qelderdelta@qelderdelta-UX331UA:~/Study/OS/os_lab_4/src$ cat 3.txt
	this should be in file3
	
	
	qelderdelta@qelderdelta-UX331UA:~/Study/OS/os_lab_1/src$ strace ./a.out
	execve("./a.out", ["./a.out"], 0x7ffda39b0c70 /* 54 vars */) = 0
	brk(NULL)                               = 0x55f985fda000
	access("/etc/ld.so.preload", R\_OK)      = -1 ENOENT (Нет такого файла или каталога)
	openat(AT\_FDCWD, "/etc/ld.so.cache", O\_RDONLY|O\_CLOEXEC) = 3
	fstat(3, \{st\_mode=S\_IFREG|0644, st\_size=142252, ...\}) = 0
	mmap(NULL, 142252, PROT\_READ, MAP\_PRIVATE, 3, 0) = 0x7fe4d59cf000
	close(3)                                = 0
	openat(AT\_FDCWD, "/lib/x86\_64-linux-gnu/libpthread.so.0", O\_RDONLY|O\_CLOEXEC) = 3
	read(3, "\textbackslash\{\}177ELF\textbackslash\{\}2\textbackslash\{\}1\textbackslash\{\}1\textbackslash\{\}3\textbackslash\{\}0\textbackslash\{\}0\textbackslash\{\}0\textbackslash\{\}0\textbackslash\{\}0\textbackslash\{\}0\textbackslash\{\}0\textbackslash\{\}0\textbackslash\{\}3\textbackslash\{\}0>\textbackslash\{\}0\textbackslash\{\}1\textbackslash\{\}0\textbackslash\{\}0\textbackslash\{\}0\textbackslash\{\}200|\textbackslash\{\}0\textbackslash\{\}0\textbackslash\{\}0\textbackslash\{\}0\textbackslash\{\}0\textbackslash\{\}0"..., 832) = 832
	fstat(3, \{st\_mode=S\_IFREG|0755, st\_size=149840, ...\}) = 0
	mmap(NULL, 8192, PROT\_READ|PROT\_WRITE, MAP\_PRIVATE|MAP\_ANONYMOUS, -1, 0) = 0x7fe4d59cd000
	mmap(NULL, 132288, PROT\_READ, MAP\_PRIVATE|MAP\_DENYWRITE, 3, 0) = 0x7fe4d59ac000
	mmap(0x7fe4d59b3000, 61440, PROT\_READ|PROT\_EXEC, MAP\_PRIVATE|MAP\_FIXED|MAP\_DENYWRITE, 3, 0x7000) = 0x7fe4d59b3000
	mmap(0x7fe4d59c2000, 20480, PROT\_READ, MAP\_PRIVATE|MAP\_FIXED|MAP\_DENYWRITE, 3, 0x16000) = 0x7fe4d59c2000
	mmap(0x7fe4d59c7000, 8192, PROT\_READ|PROT\_WRITE, MAP\_PRIVATE|MAP\_FIXED|MAP\_DENYWRITE, 3, 0x1a000) = 0x7fe4d59c7000
	mmap(0x7fe4d59c9000, 13504, PROT\_READ|PROT\_WRITE, MAP\_PRIVATE|MAP\_FIXED|MAP\_ANONYMOUS, -1, 0) = 0x7fe4d59c9000
	close(3)                                = 0
	openat(AT\_FDCWD, "/lib/x86\_64-linux-gnu/libc.so.6", O\_RDONLY|O\_CLOEXEC) = 3
	read(3, "\textbackslash\{\}177ELF\textbackslash\{\}2\textbackslash\{\}1\textbackslash\{\}1\textbackslash\{\}3\textbackslash\{\}0\textbackslash\{\}0\textbackslash\{\}0\textbackslash\{\}0\textbackslash\{\}0\textbackslash\{\}0\textbackslash\{\}0\textbackslash\{\}0\textbackslash\{\}3\textbackslash\{\}0>\textbackslash\{\}0\textbackslash\{\}1\textbackslash\{\}0\textbackslash\{\}0\textbackslash\{\}0\textbackslash\{\}200l\textbackslash\{\}2\textbackslash\{\}0\textbackslash\{\}0\textbackslash\{\}0\textbackslash\{\}0\textbackslash\{\}0"..., 832) = 832
	fstat(3, \{st\_mode=S\_IFREG|0755, st\_size=2000480, ...\}) = 0
	mmap(NULL, 2008696, PROT\_READ, MAP\_PRIVATE|MAP\_DENYWRITE, 3, 0) = 0x7fe4d57c1000
	mmap(0x7fe4d57e6000, 1519616, PROT\_READ|PROT\_EXEC, MAP\_PRIVATE|MAP\_FIXED|MAP\_DENYWRITE, 3, 0x25000) = 0x7fe4d57e6000
	mmap(0x7fe4d5959000, 299008, PROT\_READ, MAP\_PRIVATE|MAP\_FIXED|MAP\_DENYWRITE, 3, 0x198000) = 0x7fe4d5959000
	mmap(0x7fe4d59a2000, 24576, PROT\_READ|PROT\_WRITE, MAP\_PRIVATE|MAP\_FIXED|MAP\_DENYWRITE, 3, 0x1e0000) = 0x7fe4d59a2000
	mmap(0x7fe4d59a8000, 13944, PROT\_READ|PROT\_WRITE, MAP\_PRIVATE|MAP\_FIXED|MAP\_ANONYMOUS, -1, 0) = 0x7fe4d59a8000
	close(3)                                = 0
	mmap(NULL, 12288, PROT\_READ|PROT\_WRITE, MAP\_PRIVATE|MAP\_ANONYMOUS, -1, 0) = 0x7fe4d57be000
	arch\_prctl(ARCH\_SET\_FS, 0x7fe4d57be740) = 0
	mprotect(0x7fe4d59a2000, 12288, PROT\_READ) = 0
	mprotect(0x7fe4d59c7000, 4096, PROT\_READ) = 0
	mprotect(0x55f9844ee000, 4096, PROT\_READ) = 0
	mprotect(0x7fe4d5a1c000, 4096, PROT\_READ) = 0
	munmap(0x7fe4d59cf000, 142252)          = 0
	set\_tid\_address(0x7fe4d57bea10)         = 3655
	set\_robust\_list(0x7fe4d57bea20, 24)     = 0
	rt\_sigaction(SIGRTMIN, \{sa\_handler=0x7fe4d59b36c0, sa\_mask=[], sa\_flags=SA\_RESTORER|SA\_SIGINFO, sa\_restorer=0x7fe4d59bff40\}, NULL, 8) = 0
	rt\_sigaction(SIGRT\_1, \{sa\_handler=0x7fe4d59b3760, sa\_mask=[], sa\_flags=SA\_RESTORER|SA\_RESTART|SA\_SIGINFO, sa\_restorer=0x7fe4d59bff40\}, NULL, 8) = 0
	rt\_sigprocmask(SIG\_UNBLOCK, [RTMIN RT\_1], NULL, 8) = 0
	prlimit64(0, RLIMIT\_STACK, NULL, \{rlim\_cur=8192*1024, rlim\_max=RLIM64\_INFINITY\}) = 0
	read(0, 2
	"2", 1)                         = 1
	read(0, "\textbackslash\{\}n", 1)                        = 1
	brk(NULL)                               = 0x55f985fda000
	brk(0x55f985ffb000)                     = 0x55f985ffb000
	read(0, 1.txt
	"1", 1)                         = 1
	read(0, ".", 1)                         = 1
	read(0, "t", 1)                         = 1
	read(0, "x", 1)                         = 1
	read(0, "t", 1)                         = 1
	read(0, "\textbackslash\{\}n", 1)                        = 1
	read(0, 2.txt
	"2", 1)                         = 1
	read(0, ".", 1)                         = 1
	read(0, "t", 1)                         = 1
	read(0, "x", 1)                         = 1
	read(0, "t", 1)                         = 1
	read(0, "\textbackslash\{\}n", 1)                        = 1
	statfs("/dev/shm/", \{f\_type=TMPFS\_MAGIC, f\_bsize=4096, f\_blocks=1001828, f\_bfree=992030, f\_bavail=992030, f\_files=1001828, f\_ffree=1001776, f\_fsid=\{val=[0, 0]\}, f\_namelen=255, f\_frsize=4096, f\_flags=ST\_VALID|ST\_NOSUID|ST\_NODEV\}) = 0
	futex(0x7fe4d59cc3b0, FUTEX\_WAKE\_PRIVATE, 2147483647) = 0
	openat(AT\_FDCWD, "/dev/shm/sem.semaphore0input", O\_RDWR|O\_NOFOLLOW) = -1 ENOENT (Нет такого файла или каталога)
	getpid()                                = 3655
	lstat("/dev/shm/iHu5Yh", 0x7ffd496ec340) = -1 ENOENT (Нет такого файла или каталога)
	openat(AT\_FDCWD, "/dev/shm/iHu5Yh", O\_RDWR|O\_CREAT|O\_EXCL, 01411) = 3
	write(3, "\textbackslash\{\}0\textbackslash\{\}0\textbackslash\{\}0\textbackslash\{\}0\textbackslash\{\}0\textbackslash\{\}0\textbackslash\{\}0\textbackslash\{\}0\textbackslash\{\}200\textbackslash\{\}0\textbackslash\{\}0\textbackslash\{\}0\textbackslash\{\}0\textbackslash\{\}0\textbackslash\{\}0\textbackslash\{\}0\textbackslash\{\}0\textbackslash\{\}0\textbackslash\{\}0\textbackslash\{\}0\textbackslash\{\}0\textbackslash\{\}0\textbackslash\{\}0\textbackslash\{\}0\textbackslash\{\}0\textbackslash\{\}0\textbackslash\{\}0\textbackslash\{\}0\textbackslash\{\}0\textbackslash\{\}0\textbackslash\{\}0\textbackslash\{\}0", 32) = 32
	mmap(NULL, 32, PROT\_READ|PROT\_WRITE, MAP\_SHARED, 3, 0) = 0x7fe4d59f1000
	link("/dev/shm/iHu5Yh", "/dev/shm/sem.semaphore0input") = 0
	fstat(3, \{st\_mode=S\_IFREG|S\_ISVTX|0411, st\_size=32, ...\}) = 0
	unlink("/dev/shm/iHu5Yh")               = 0
	close(3)                                = 0
	openat(AT\_FDCWD, "/dev/shm/sem.semaphore0output", O\_RDWR|O\_NOFOLLOW) = -1 ENOENT (Нет такого файла или каталога)
	getpid()                                = 3655
	lstat("/dev/shm/CYRcbL", 0x7ffd496ec340) = -1 ENOENT (Нет такого файла или каталога)
	openat(AT\_FDCWD, "/dev/shm/CYRcbL", O\_RDWR|O\_CREAT|O\_EXCL, 01411) = 3
	write(3, "\textbackslash\{\}1\textbackslash\{\}0\textbackslash\{\}0\textbackslash\{\}0\textbackslash\{\}0\textbackslash\{\}0\textbackslash\{\}0\textbackslash\{\}0\textbackslash\{\}200\textbackslash\{\}0\textbackslash\{\}0\textbackslash\{\}0\textbackslash\{\}0\textbackslash\{\}0\textbackslash\{\}0\textbackslash\{\}0\textbackslash\{\}0\textbackslash\{\}0\textbackslash\{\}0\textbackslash\{\}0\textbackslash\{\}0\textbackslash\{\}0\textbackslash\{\}0\textbackslash\{\}0\textbackslash\{\}0\textbackslash\{\}0\textbackslash\{\}0\textbackslash\{\}0\textbackslash\{\}0\textbackslash\{\}0\textbackslash\{\}0\textbackslash\{\}0", 32) = 32
	mmap(NULL, 32, PROT\_READ|PROT\_WRITE, MAP\_SHARED, 3, 0) = 0x7fe4d59f0000
	link("/dev/shm/CYRcbL", "/dev/shm/sem.semaphore0output") = 0
	fstat(3, \{st\_mode=S\_IFREG|S\_ISVTX|0411, st\_size=32, ...\}) = 0
	unlink("/dev/shm/CYRcbL")               = 0
	close(3)                                = 0
	unlink("/dev/shm/sem.semaphore0input")  = 0
	unlink("/dev/shm/sem.semaphore0output") = 0
	openat(AT\_FDCWD, "/dev/shm/sem.semaphore1input", O\_RDWR|O\_NOFOLLOW) = -1 ENOENT (Нет такого файла или каталога)
	getpid()                                = 3655
	lstat("/dev/shm/U0Srne", 0x7ffd496ec340) = -1 ENOENT (Нет такого файла или каталога)
	openat(AT\_FDCWD, "/dev/shm/U0Srne", O\_RDWR|O\_CREAT|O\_EXCL, 01411) = 3
	write(3, "\textbackslash\{\}0\textbackslash\{\}0\textbackslash\{\}0\textbackslash\{\}0\textbackslash\{\}0\textbackslash\{\}0\textbackslash\{\}0\textbackslash\{\}0\textbackslash\{\}200\textbackslash\{\}0\textbackslash\{\}0\textbackslash\{\}0\textbackslash\{\}0\textbackslash\{\}0\textbackslash\{\}0\textbackslash\{\}0\textbackslash\{\}0\textbackslash\{\}0\textbackslash\{\}0\textbackslash\{\}0\textbackslash\{\}0\textbackslash\{\}0\textbackslash\{\}0\textbackslash\{\}0\textbackslash\{\}0\textbackslash\{\}0\textbackslash\{\}0\textbackslash\{\}0\textbackslash\{\}0\textbackslash\{\}0\textbackslash\{\}0\textbackslash\{\}0", 32) = 32
	mmap(NULL, 32, PROT\_READ|PROT\_WRITE, MAP\_SHARED, 3, 0) = 0x7fe4d59ef000
	link("/dev/shm/U0Srne", "/dev/shm/sem.semaphore1input") = 0
	fstat(3, \{st\_mode=S\_IFREG|S\_ISVTX|0411, st\_size=32, ...\}) = 0
	unlink("/dev/shm/U0Srne")               = 0
	close(3)                                = 0
	openat(AT\_FDCWD, "/dev/shm/sem.semaphore1output", O\_RDWR|O\_NOFOLLOW) = -1 ENOENT (Нет такого файла или каталога)
	getpid()                                = 3655
	lstat("/dev/shm/rdhNzH", 0x7ffd496ec340) = -1 ENOENT (Нет такого файла или каталога)
	openat(AT\_FDCWD, "/dev/shm/rdhNzH", O\_RDWR|O\_CREAT|O\_EXCL, 01411) = 3
	write(3, "\textbackslash\{\}1\textbackslash\{\}0\textbackslash\{\}0\textbackslash\{\}0\textbackslash\{\}0\textbackslash\{\}0\textbackslash\{\}0\textbackslash\{\}0\textbackslash\{\}200\textbackslash\{\}0\textbackslash\{\}0\textbackslash\{\}0\textbackslash\{\}0\textbackslash\{\}0\textbackslash\{\}0\textbackslash\{\}0\textbackslash\{\}0\textbackslash\{\}0\textbackslash\{\}0\textbackslash\{\}0\textbackslash\{\}0\textbackslash\{\}0\textbackslash\{\}0\textbackslash\{\}0\textbackslash\{\}0\textbackslash\{\}0\textbackslash\{\}0\textbackslash\{\}0\textbackslash\{\}0\textbackslash\{\}0\textbackslash\{\}0\textbackslash\{\}0", 32) = 32
	mmap(NULL, 32, PROT\_READ|PROT\_WRITE, MAP\_SHARED, 3, 0) = 0x7fe4d59ee000
	link("/dev/shm/rdhNzH", "/dev/shm/sem.semaphore1output") = 0
	fstat(3, \{st\_mode=S\_IFREG|S\_ISVTX|0411, st\_size=32, ...\}) = 0
	unlink("/dev/shm/rdhNzH")               = 0
	close(3)                                = 0
	unlink("/dev/shm/sem.semaphore1input")  = 0
	unlink("/dev/shm/sem.semaphore1output") = 0
	getpid()                                = 3655
	openat(AT\_FDCWD, "/tmp/tmp\_file.uZNfMa", O\_RDWR|O\_CREAT|O\_EXCL, 0600) = 3
	write(3, "   ", 3)                      = 3
	mmap(NULL, 3, PROT\_READ|PROT\_WRITE, MAP\_SHARED, 3, 0) = 0x7fe4d59ed000
	clone(child\_stack=NULL, flags=CLONE\_CHILD\_CLEARTID|CLONE\_CHILD\_SETTID|SIGCHLD, child\_tidptr=0x7fe4d57bea10) = 3656
	clone(child\_stack=NULL, flags=CLONE\_CHILD\_CLEARTID|CLONE\_CHILD\_SETTID|SIGCHLD, child\_tidptr=0x7fe4d57bea10) = 3657
	read(0, xxxddd
	"x", 1)                         = 1
	futex(0x7fe4d59f1000, FUTEX\_WAKE, 1)    = 1
	read(0, "x", 1)                         = 1
	futex(0x7fe4d59ef000, FUTEX\_WAKE, 1)    = 1
	read(0, "x", 1)                         = 1
	futex(0x7fe4d59f1000, FUTEX\_WAKE, 1)    = 1
	read(0, "d", 1)                         = 1
	futex(0x7fe4d59ef000, FUTEX\_WAKE, 1)    = 1
	read(0, "d", 1)                         = 1
	futex(0x7fe4d59f1000, FUTEX\_WAKE, 1)    = 1
	read(0, "d", 1)                         = 1
	futex(0x7fe4d59ef000, FUTEX\_WAKE, 1)    = 1
	read(0, "\textbackslash\{\}n", 1)                        = 1
	futex(0x7fe4d59f1000, FUTEX\_WAKE, 1)    = 1
	read(0, "", 1)                          = 0
	futex(0x7fe4d59f1000, FUTEX\_WAKE, 1)    = 1
	futex(0x7fe4d59ef000, FUTEX\_WAKE, 1)    = 1
	--- SIGCHLD \{si\_signo=SIGCHLD, si\_code=CLD\_EXITED, si\_pid=3656, si\_uid=1000, si\_status=0, si\_utime=0, si\_stime=0\} ---
	close(3)                                = 0
	munmap(0x7fe4d59ed000, 3)               = 0
	--- SIGCHLD \{si\_signo=SIGCHLD, si\_code=CLD\_EXITED, si\_pid=3657, si\_uid=1000, si\_status=0, si\_utime=0, si\_stime=0\} ---
	exit\_group(0)                           = ?
	+++ exited with 0 +++

\end{verbatim}
\newpage
\section{Выводы}

В ходе выполнения лабораторной работы я в очередной раз воспользовался утилитой strace для того, чтобы наглядно увидеть какие системные вызовы были совершены в ходе работы моей программы.
\end{document}